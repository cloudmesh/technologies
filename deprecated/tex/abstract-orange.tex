\section{Orange}
\index{Orange}

Orange~\cite{hid-sp18-504-orange} is a  data mining, visualization,
and machine learning toolkit based on Python 3. This software is
developed in such a way as to allow practitioners to have varying
degrees of technical background (including complete novices) and
still utilize the product's capabilities~\cite{hid-sp18-504-orange},
Orange has an interactive data visualization interface which allows
for a simpler approach to perform complex data mining and machine
learning practices and to derive insightful knowledge. Orange also
provides a visual programming component in the form of widgets to
perform qualitative analysis through a visualized workflow map. Depending
on a widget's function, it is then grouped into a class, encouraging
the use of various widgets in a typical given workflow. These widget
visualizations also assist in the communication of analytic processes
between domain experts and data scientists, which has encouraged the
use of this product in academic and research settings~\cite{hid-sp18-504-orange};
particularly with domains involving `biomedicine, bioinformatics, [and]
genomic research'~\cite{hid-sp18-504-wiki-orange}.

This open source toolkit's latest version (3+) uses various python
libraries for computation, while utilizing the Qt framework for the
visualization end~\cite{hid-sp18-504-wiki-orange}. The available Python
classes and methods include classes based on data models, preprocessing,
classification, regression, clustering, distance, evaluation, and projection.
The classification and regression classes offer the largest number of
available methods, such as random forests, Naive Bayes, neural networks,
and k-nearest neighbors~\cite{hid-sp18-504-orange}. These classes can
either be used directly as a Python library, or used in Orange's widget
sets. It is also possible to create custom widgets and include them in an
Orange workflow~\cite{hid-sp18-504-wiki-orange}.
