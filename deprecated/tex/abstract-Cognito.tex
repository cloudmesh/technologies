\section{Cognito}
\index{Cognito}
\index{AWS Cognito}

The AWS Cognito~\cite{hid-sp18-518-Cognito} service is used to federate your
user registration and their ability to sign into your services. The solution
allows you to easily manage user pools and can integrate with multiple SDKs 
like Java, Python, PHP and Ruby. The client application can be configured to
use SAML, OIDC or other backend user directory services. The service is 
intended to be used in conjuction with AWS IAM and STS.


% Notes

% Note that this section has references missing such as tacc-stat. File
% names must be lower case and not contain an underscore. The abstract
% is contained in a file called abstract-<tech>.tex. The bib file is
% contained in a file called <hid>.bib.

% Make sure that bib labels have the prefix of your hid. In our case it
% is something like hid-sp18-999. Make sure you do not under any
% circumstances use underscores in bib labels as they break our scripts.

% Make sure you resolve bibtex warnings and errors.

% Make sure to use the Makefile (and modify it accordingly) to check if
% your latex file compiles. Only check it into git if it compiles. If
% you do not know how to use Makefiles, please lear nit or use alternative
% commands in the terminal. Look at the Makefile in which order you
% need to execute them if you do not use makefiles. 

% UNDER NO CIRCUMSTANCES COMMIT THE GENERATED PDF INTO GITHUB. COMMIT
% EVERY FILE INDIVIDUALLY TO MAKE SURE YOU AVOID THIS. WE WILL
% DEDUCT YOU POINTS IF (a) YOU COMMITTED A PDF FILE (b) YOUR LATEX FILE
% DOES NOT COMPILE OR CONTAINS ERRORS (c) YOUR BIB FILE IS INCOMPLETE OR
% CONTAINS ERRORS. (d) YOUR TEXT IS NOT FORMATTED TO HAVE A MAXIMUM OF
% 80 CHARACTERS IN EACH LINE. 

% Points in case of a, b, c you will get 0 points as you will cause our
% scripts to break. In case of d you will get a 50\% point deduction. We
% want to set with this simple example a mechanism for you to check
% larger papers. It is not sufficient to say but my paper compiles in
% sharelatex. It is your responsibility to make sure that what is in your
% directory compiles properly in LaTeX. You are allowed to use a native
% LaTeX deployment if you have one set up. Make sure to install ALL of
% latex and not just the reduced version. 

