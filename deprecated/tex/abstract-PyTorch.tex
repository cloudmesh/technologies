\section{PyTorch}
\index{PyTorch}

``PyTorch is a open source python package that has high level features of 
Tensor computation with strong GPU acceleration and Deep 
Neural Networks built on a tape-based autograd system''
~\cite{hid-sp18-520-PyTorch}.
PyTorch has many packages and are used for deep learning, multi processing,
loading data. It is fast and has high computation speed when run with any size 
of datasets.
Out of many libraries of PyTorch, ``A PyTorch Tensor is conceptually identical 
to a numpy array: a Tensor is an n-dimensional array, and PyTorch provides many
functions for operating on these Tensors. Like numpy arrays, PyTorch Tensors do 
not know anything about deep learning or computational graphs or gradients; they
are a generic tool for scientific computing''~\cite{hid-sp18-520-PyTorchtensor}.
PyTorch supports dynamic computation graphs, where the computational graph can 
be created in real run time.
``Respect to Grad, This is especially useful when you want to freeze part of your
model, or you know in advance that you are not going to use gradients w.r.t. 
some parameters. If there is a single input to an operation that requires 
gradient, its output will also require gradient. Conversely, only if all inputs 
do not require gradient, the output also will not require it. Backward 
computation is never performed in the subgraphs, where all Variables did not 
require gradients''~\cite{hid-sp18-520-PyTorchgrad}.
