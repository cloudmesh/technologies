\section{Redis}
\index{Redis}

``Redis is an open source (BSD licensed), in-memory data structure store, used
as a database, cache and message broker. It supports data structures such as
strings, hashes, lists, sets, sorted sets with range queries, bitmaps, 
hyperloglogs and geospatial indexes with radius queries. Redis has built-in 
replication, Lua scripting, LRU eviction, transactions and different levels 
of on-disk persistence, and provides high availability via Redis Sentinel and
automatic partitioning with Redis Cluster.
It can run atomic operations on these types, like appending to a string, 
incrementing the value in a hash; pushing an element to a list; computing set
intersection, union and difference, or getting the member with highest 
ranking in a sorted set.
In order to achieve its outstanding performance, Redis works with an in-memory
dataset. Depending on your use case, you can persist it either by dumping the
dataset to disk every once in a while, or by appending each command to a log.
Redis also supports trivial-to-setup master-slave asynchronous replication, 
with very fast non-blocking first synchronization, auto-reconnection with 
partial resynchronization on net split''~\cite{hid-sp18-520-Redis}.
Redis is No SQL database, supports Key value databases by mapping its key
to type of values.
