\section{Amazon Elastic Beanstalk}

AWS Elastic Beanstalk~\cite{hid-sp18-420-amazon-elastic-beanstalk} is a managed
service used for application deployment and management. Using EBS it is easy to
quickly deploy and manage applications in the AWS Cloud. Developers simply
upload their application, and Elastic Beanstalk automatically handles the
deployment details of capacity provisioning, load balancing, auto-scaling, and
application health monitoring~\cite{hid-sp18-420-amazon-elastic-beanstalk-FAQ}.

Elastic Beanstalk supports applications which are developed in Java, PHP, \.NET,
Node.js, Python, and Ruby as well as different container types for each
language. A container is used to define the infrastructure and technology stack
to be used for a given
environment~\cite{hid-sp18-420-amazon-elastic-beanstalk-FAQ}. AWS Elastic
Beanstalk runs on the Amazon Linux AMI and the Windows Server 2012 R2 AMI
provided by Amazon. Initially, it takes some time to create AWS resources
required to run the application. User then can have multiple versions of their
applications running at the same time. Hence, user can create different
environments such as staging and production where each environment runs with its
own configurations and resources. AWS Elastic Beanstalk does not have any extra
charges. Users need to pay for the resources they have used to store and run the
applications such as EC2, S3, RDS or any other resources used.
